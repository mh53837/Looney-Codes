\chapter{Arhitektura i dizajn sustava}

\begin{figure}[H]
	\includegraphics[scale=1]{slike/prikaz_arhitekture.png}
	\centering
	\caption{Prikaz arhitekture sustava}
\end{figure}

Arhitektura sustava može se podijeliti na tri glavna podsustava, a to su \textbf{Frontend Web aplikacija, Backend Web aplikacija i baza podataka}.
\begin{itemize}
	\item 	\textit{\textbf{Web poslužitelj}} prima zahtjeve od klijenata putem interneta, obrađuje ih i pruža resurse poput web stranice, slike, videa i datoteke kao odgovor. Za distribuciju resursa najčešće se koriste protokoli kao što su HTTP (Hypertext Transfer Protocol) ili HTTPS (HTTP Secure).
	\item 	\textit{\textbf{Web aplikacija}} je program koji se izvršava na web pregledniku (Google Chrome, Mozilla Firefox, Safari itd.) i pruža korisnicima mogućnost izvršavanja željenih zahtjeva, odnosno interakciju s određenim uslugama i funkcionalnostima web aplikacije. Prilikom obrade zahtjeva pristupa se bazi podataka i korisniku se odgovor vraća kao HTML dokument.
	\item 	\textit{\textbf{Baza podataka}} je organizirani skup podataka namijenjen za efikasno upravljanje, ažuriranje, pretraživanje i dohvat podataka. Uloge baze podataka u web aplikaciji su pohrana podataka, očuvanje integriteta podataka i osiguravanje dosljednosti, upravljanje transakcijama itd.
\end{itemize}

Za izradu naše web aplikacije odabrali smo \textit{\textbf{Spring Boot}} (open-source Java framework) i \textit{\textbf{React}} (open-source JavaScript library). Odabrana razvojna okruženja su IntelliJ IDEA i Eclipse IDE za Spring Boot, odnosno Visual Studio Code i WebStorm za React. Za izradu baze podatka koristimo PostgreSQL.
Arhitektura, koja je podržana Spring Boot-om, temelji se na \textit{\textbf{MVC (Model-View-Controller)}} konceptu koji strogo odvaja model, akcije i prezentaciju, olakšava razvoj i održavanje aplikacije te čini aplikaciju prilagodljivom i jednostavnom za proširenje.
\begin{itemize}
	\item \textit{\textbf{Model}} - predstavlja poslovnu logiku, odnosno dinamičke strukture podataka, mijenja pogled na zahtjev kontrolera. Modeli u pravilu predstavljaju podatke(objekte) koje aplikacija obrađuje.
	\item \textit{\textbf{View}} - ono što klijent vidi, odnosno korisničko sučelje potrebno za interakciju s aplikacijom kao što su dijagrami, linkovi, slike, tablice itd.
	\item \textit{\textbf{Controller}} - presreće zahtjeve klijenata i prilagođava model, odnosno obavještava model o promjeni zahtjeva korisnika i u skladu sa zahtjevima daje prikladan View.
\end{itemize}

\section{Baza podataka}
			
Za potrebe našeg sustava koristit ćemo relacijsku bazu podataka koja svojom strukturom olakšava modeliranje stvarnog svijeta. Gradivna jedinka baze je relacija, od- nosno tablica koja je definirana svojim imenom i skupom atributa. Zadaća baze podataka je brza i jednostavna pohrana, izmjena i dohvat podataka za daljnju obradu. Baza podataka ove aplikacije sastoji se od sljedećih entiteta:
\begin{itemize}
	\item Natjecanje
	\item VirtualnoNatjecanje
	\item Korisnik
	\item Pehar
	\item Zadatak
	\item Rješenje
	\item TestniPrimjer		
\end{itemize}
				
\subsection{Opis tablica}
			

\textbf{Natjecanje} \quad Ovaj entitet sadržava informacije o natjecanju koje trenutno rješava korisnik. Atributi koje sadržava su: natjecanjeID, korisnikID, nazivNatjecanja, pocetakNatjecanja, krajNatjecanja. Ovaj entitet ima  \textit{One-to-Many} vezu s entitetom Pehar preko atributa natjecanjeID, te  \textit{One-to-Many} vezu sa slabim entitetom VirtualnoNatjecanje preko atributa originalnoNatjecanjeID. Ima  \textit{One-to-Many} vezu s entitetom Korisnik preko atributa korisnikID.
				
				
				\begin{longtblr}[
					label=none,
					entry=none
					]{
						width = \textwidth,
						colspec={|X[9,l]|X[6, l]|X[18, l]|}, 
						rowhead = 1,
					} %definicija širine tablice, širine stupaca, poravnanje i broja redaka naslova tablice
					\hline \SetCell[c=3]{c}{\textbf{Natjecanje}}	 \\ \hline[3pt]
					\SetCell{LightGreen}natjecanjeID & INT	&  	Jedinstveni identifikator natjecanja  	\\ \hline
					\SetCell{LightBlue}korisnikID	& INT &  Jedinstveni identifikator korisnika 	\\ \hline
					nazivNatjecanja & VARCHAR &  Naziv natjecanja \\ \hline 
					pocetakNatjecanja & TIMESTAMP	&  	Vrijeme početka natjecanja	\\ \hline 
					krajNatjecanja	& TIMESTAMP &   Vrijeme završetka natjecanja	\\ \hline
				\end{longtblr}
				
\textbf{VirtualnoNatjecanje} \quad Ovaj entitet sadržava informacije o virtualnom natjecanju koje je pokrenuo korisnik na temelju nekog natjecanja. Atributi koje sadržava su: virtualnoNatjecanjeID, korisnikID, originalnoNatjecanjeID i pocetakNatjecanja. Ovaj entitet ima \textit{Many-To-One} vezu s entitetom Korisnik preko atributa i vanjskog ključa korisnikID. Atribut originalnoNatjecanjeID predstavlja vanjski ključ koji se referencira na natjecanjeID u relaciji Natjecanje pa se time tvori \textit{Many-To-One} veza.
		
				\begin{longtblr}[
					label=none,
					entry=none
					]{
						width = \textwidth,
						colspec={|X[12,l]|X[7, l]|X[20, l]|},
						rowhead = 1,
					} %definicija širine tablice, širine stupaca, poravnanje i broja redaka naslova tablice
					\hline \SetCell[c=3]{c}{\textbf{VirtualnoNatjecanje}}	 \\ \hline[3pt]
					\SetCell{LightGreen}virtualnoNatjecanjeID & INT & Jedinstveni identifikator virtualnog natjecanja  	\\ \hline
					\SetCell{LightBlue}korisnikID & INT &  ID korisnika koji je stvorio virtualno natjecanje \\ \hline
					\SetCell{LightBlue}originalnoNatjecanjeID & INT &  ID natjecanja na kojem se temelji virtualno natjecanje 	\\ \hline
					pocetakNatjecanja & TIMESTAMP & Vrijeme početka natjecanja 	\\ \hline
				\end{longtblr}

\textbf{Korisnik} \quad Ovaj entitet sadržava sve bitne informacije o korisniku aplikacije. Sadrži atribute: korisnickoIme, lozinka, ime, prezime, email, fotografija, vrijemeRegistracije, ulogaID, requestedUloga i confirmedEmail. Ovaj entitet je u vezi \textit{One-to-Many} s entitetom Natjecanje preko atributa korisnikID, u vezi \textit{One-to-Many} s entitetom VirtualnoNatjecanje preko atributa korisnikID, u vezi  \textit{One-to-Many} s entitetom Zadatak preko atributa voditeljID, u vezi  \textit{One-to-Many} s entitetom Rjesenje preko atributa natjecateljID te je u vezi \textit{One-to-Many} s entitetom Pehar preko atributa natjecateljID.
				
				\begin{longtblr}[
					label=none,
					entry=none
					]{
						width = \textwidth,
						colspec={|X[9,l]|X[6, l]|X[18, l]|}, 
						rowhead = 1,
					} %definicija širine tablice, širine stupaca, poravnanje i broja redaka naslova tablice
					\hline \SetCell[c=3]{c}{\textbf{Korisnik}}	 \\ \hline[3pt]
					\SetCell{LightGreen}korisnikID & INT	&  	Jedinstveni identifikator korisnika  	\\ \hline
					korisnickoIme	& VARCHAR &  Jedinstveno ime korisnika 	\\ \hline 
					lozinka & VARCHAR &  Korisnikova lozinka \\ \hline 
					ime & VARCHAR	&  	Ime korisnika	\\ \hline 
					prezime	& VARCHAR &   Prezime korisnika	\\ \hline 
					email & VARCHAR & Elektronička pošta korisnika \\ \hline 
					fotografija & PATH & Fotografija korisnika \\ \hline 
					vrijemeRegistracije & TIMESTAMP & Vrijeme kada se korisnik registrirao u sustav \\ \hline 
					ulogaID & VARCHAR & Jedinstveni identifikator uloge	\\ \hline
					requestedUloga & VARCHAR & Uloga koju je korisnik zatražio \\ \hline
					confirmedEmail & BOOLEAN & Zastavica koja određuje je li korisnik potvrdio email za registraciju	\\ \hline
				\end{longtblr}
				
\textbf{Pehar} \quad Ovaj entitet predstavlja pehar kojeg natjecatelji (korisnici) mogu osvojiti u natjecanju. Sadrži atribute: peharID, natjecateljID, natjecanjeID, mjesto te slikaPehara. Ovaj entitet ima  \textit{Many-to-One} vezu s entitetom Natjecanje preko atributa natjecanjeID, te ima vezu  \textit{Many-to-One} s entitetom Korisnik preko atributa natjecateljID.
				
				\begin{longtblr}[
					label=none,
					entry=none
					]{
						width = \textwidth,
						colspec={|X[9,l]|X[7, l]|X[20, l]|},
						rowhead = 1,
					} %definicija širine tablice, širine stupaca, poravnanje i broja redaka naslova tablice
					\hline \SetCell[c=3]{c}{\textbf{Pehar}}	 \\ \hline[3pt]
					\SetCell{LightGreen}peharID & INT	&  	Jedinstveni identifikator pehara  	\\ \hline
					\SetCell{LightBlue}natjecateljID	 & INT &  Jedinstveni identifikator natjecatelja 	\\ \hline 
					\SetCell{LightBlue}natjecanjeID & INT &  Jedinstveni identifikator natjecanja	\\ \hline 
					mjesto & INT & Mjesto koje je dobiveno peharom (1, 2 ili 3) \\ \hline
					slikaPehara & VARCHAR & Slika dobivenog pehara \\ \hline
				\end{longtblr}

\textbf{Zadatak} \quad Ovaj entitet sadržava sve bitne značajke za definiciju jednog zadataka u aplikaciji. Atributi koje sadržava su: zadatakID, voditeljID, nazivZadatka, brojBodova, vremenskoOgranicenje, tekstZadatka, tezinaZadatka te privatniZadatak. Ovaj entitet ima \textit{One-To-Many} vezu sa slabim entitetom TestniPrimjer preko atributa zadatakID. Vanjskim ključem natjecanjeID stvorena je opcionalna \textit{Many-To-One} veza s entitetom Natjecanje. Postoji i \textit{One-To-Many} veza s entitetom Rješenje preko atributa zadatakID. \textit{Many-To-One} veza postoji i s entitetom Korisnik preko vanjskog ključa voditeljID (označava identifikator korisnika s ulogom voditelja).
				
				\begin{longtblr}[
					label=none,
					entry=none
					]{
						width = \textwidth,
						colspec={|X[11,l]|X[6, l]|X[18, l]|},
						rowhead = 1,
					} %definicija širine tablice, širine stupaca, poravnanje i broja redaka naslova tablice
					\hline \SetCell[c=3]{c}{\textbf{Zadatak}}	 \\ \hline[3pt]
					\SetCell{LightGreen}zadatakID & INT	&  	Jedinstveni  privatni identifikator zadatka  	\\ \hline
					nazivZadatka & VARCHAR & Naziv zadatka \\ \hline
					tekstZadatka & VARCHAR & Tekst kojim je zadan zadatak \\ \hline
					tezinaZadatka & VARCHAR & Predstavlja težinu zadatka \\ \hline
					brojBodova & INT & Broj bodova koje nosi zadatak \\ \hline
					vremenskoOgranicenje & INT & Vremensko ograničenje za izvođenje predanog rješenja \\ \hline
					privatniZadatak & BOOLEAN & Zastavica koja određuje je li zadatak privatan \\ \hline
					\SetCell{LightBlue}VoditeljID & INT &  Jedinstveni identifikator voditelja koji je stvorio zadatak	\\ \hline 
				\end{longtblr}
				
				
\textbf{Rješenje} \quad Ovaj slabi entitet sadržava informacije o predanim rješenjima pojedinog korisnika za određeni zadatak. Atributi koje sadržava su: rješenjeRb, natjecateljID, zadatakID, vrijemeOdgovora, brojTocnihPrimjera, brojBodova i programskiKod. Ovaj entitet ima \textit{Many-To-One} vezu s  entitetom Zadatak preko atributa i vanjskog ključa zadatakID. \textit{Many-To-One} veza postoji i s entitetom Korisnik preko vanjskog ključa korisnikID.
				
				\begin{longtblr}[
					label=none,
					entry=none
					]{
						width = \textwidth,
						colspec={|X[10,l]|X[6, l]|X[18, l]|}, 
						rowhead = 1,
					} %definicija širine tablice, širine stupaca, poravnanje i broja redaka naslova tablice
					\hline \SetCell[c=3]{c}{\textbf{Rješenje}}	 \\ \hline[3pt]
					\SetCell{LightGreen}rjesenjeRb & INT & Redni broj predanog rješenja određenog korisnika za predani zadatak \\ \hline
					\SetCell{LightGreen}zadatakID & INT	&  	Jedinstveni  privatni identifikator zadatka  	\\ \hline
					\SetCell{LightGreen}natjecateljID & INT &  Jedinstveni identifikator natjecatelja koji je predao rješenje	\\ \hline
					vrijemeOdgovora & TIMESTAMP & Vrijeme predaje rješenja \\ \hline
					brojTocnihOdgovora & DOUBLE & Broj primjera koji prolaze evaluaciju \\ \hline
					brojBodova & INT & Ostvareni broj bodova na zadatku \\ \hline
					programskiKod & TEXT & Programski kod predanog rješenja \\ \hline
				\end{longtblr}
				
\textbf{TestniPrimjer} \quad Ovaj slabi entitet sadržava informacije o testnim primjerima za određeni zadatak. Atributi koje sadržava su: testniPrimjerRb, zadatakID, ulazniPodaci i izlazniPodaci. Ovaj entitet ima \textit{Many-To-One} vezu s entitetom Zadatak preko atributa i vanjskog ključa zadatakID.
	
				\begin{longtblr}[
					label=none,
					entry=none
					]{
						width = \textwidth,
						colspec={|X[11,l]|X[7, l]|X[20, l]|},
						rowhead = 1,
					} %definicija širine tablice, širine stupaca, poravnanje i broja redaka naslova tablice
					\hline \SetCell[c=3]{c}{\textbf{TestniPrimjer}}	 \\ \hline[3pt]
					\SetCell{LightGreen}testniPrimjerRB & INT & Redni broj testnog primjera za pojedini zadatak  	\\ \hline
					\SetCell{LightGreen}zadatakID & INT &  Jedinstveni privatni identifikator zadatka \\ \hline
					ulazniPodaci & VARCHAR & ulazni podaci za testiranje programskog rješenja \\ \hline
					izlazniPodaci & VARCHAR & očekivani ispis programskog rješenja \\ \hline
				\end{longtblr}

\textbf{NadmetanjeZadaci} \quad Ovaj slabi entitet sadržava informacije o zadacima vezanim uz natjecanje. Atributi koje sadržava su: nadmetanjeID i zadatakID. Ovaj entitet ima \textit{Many-To-One} vezu s entitetom Zadatak preko atributa i vanjskog ključa zadatakID.
\begin{longtblr}[
					label=none,
					entry=none
				]{
					width = \textwidth,
					colspec={|X[11,l]|X[3, l]|X[20, l]|},
					rowhead = 1,
				} %definicija širine tablice, širine stupaca, poravnanje i broja redaka naslova tablice
					\hline \SetCell[c=3]{c}{\textbf{NadmetanjeZadaci}}	 \\ \hline[3pt]
					\SetCell{LightGreen}nadmetanjeID & INT & Jedinstveni privatni identifikator zadatka nadmetanja  	\\ \hline
					\SetCell{LightGreen}zadatakID & INT &  Jedinstveni privatni identifikator zadatka \\ \hline
				\end{longtblr}


\pagebreak
\subsection{Dijagram baze podataka}
%\textit{ U ovom potpoglavlju potrebno je umetnuti dijagram baze podataka. Primarni i strani ključevi moraju biti označeni, a tablice povezane. Bazu podataka je potrebno normalizirati. Podsjetite se kolegija "Baze podataka".}
\begin{figure}[H]
	\includegraphics[scale=0.7]{dijagrami/dijagram_baze_podataka.png}
	\centering
	\caption{Dijagram baze podataka}
	\label{fig:bazaPodataka} 
\end{figure}

\eject


\section{Dijagram razreda}

Na slikama \ref{fig:dijagramRazreda1} i \ref{fig:dijagramRazreda2} prikazani su dijagrami razreda koji predstavljaju backend dio arhitekture.
Prva slika opisuje servise, odnosno njihovu implementaciju i prikazani su atributi i metode u pojedinim razredima.
Servisi predstavljaju glavnu logiku i oni su uglavnom u interakciji s repozitorijima koji su između ostalog prikazani na slici \ref{fig:dijagramRazreda2}.
Osim repozitorija, na slici \ref{fig:dijagramRazreda2} prikazani su i modeli (predstavljaju strukturu baze podataka) te kontroleri. 
Kontroleri su zaduženi za upravljanje HTTP zahtjevima i pružanje prikladnog odgovora. Na drugoj slici možemo uočiti atribute i međusobne odnose navedenih razreda. \\

Razred Korisnik predstavlja registriranog korisnika koji se može registrirati kao voditelj ili natjecatelj unoseći svoje podatke u sustav. Nadmetanje je bazni razred kojeg nasljeđuju razredi Natjecanje i VirtualnoNatjecanje.
Razred Natjecanje definira naziv, početak i kraj natjecanja te voditelja, a VirtualnoNatjecanje predstavlja natjecanje koje se može naknadno pokrenuti i rješavati. Razred Pehar je entitet u kojeg spremamo koje mjesto je osvojio natjecatelj na nekom natjecanju i sliku pehara.
Razred Uloga je enumeracija koja definira ulogu korisnika (admin, natjecatelj ili voditelj). Razred Zadatak je entitet koji predstavlja određeni problem na natjecanju i sadrži referencu na testne primjere.
TestniPrimjer sadrži informacije o testnim primjerima za pojedinačni zadatak. Razred Rjesenje predstavlja predano rješenje pojedinog korisnika za zadatak. 

\begin{figure}[H]
	\includegraphics[scale=0.13]{dijagrami/serviceDiagram.png}
	\centering
	\caption{Dijagram razreda - Servisi}
	\label{fig:dijagramRazreda1}
\end{figure}

\begin{figure}[H]
	\includegraphics[scale=0.15]{dijagrami/apiDiagram.png}
	\centering
	\caption{Dijagram razreda - Kontroleri, Repozitoriji i Modeli}
	\label{fig:dijagramRazreda2}
\end{figure}

%\textbf{\textit{dio 2. revizije}}\\
%
%\textit{Prilikom druge predaje projekta dijagram razreda i opisi moraju odgovarati stvarnom stanju implementacije}
%
%
%
\eject
%
\section{Dijagram stanja}


%\textbf{\textit{dio 2. revizije}}\\
%
%\textit{Potrebno je priložiti dijagram stanja i opisati ga. Dovoljan je jedan dijagram stanja koji prikazuje \textbf{značajan dio funkcionalnosti} sustava. Na primjer, stanja korisničkog sučelja i tijek korištenja neke ključne funkcionalnosti jesu značajan dio sustava, a registracija i prijava nisu. }

Dijagram stanja prikazuje stanja objekta i prijelaze iz jednog stanja u drugo temeljene na događajima. Na slici je prikazan dijagram stanja za prijavljenog korisnika. Nakon prijave, klijentu se prikazuje početna stranica na kojoj može vidjeti kalendar sa svim natjecanjima. Korisnik može sudjelovati u natjecanju. Za odabrano natjecanje ima opciju pokretanja natjecanja, te nakon završetka natjecanja pregled rezultata i rješenja zadataka. Također, korisnik može pregledati sve prijavljene korisnike klikom na "Korisnici" i sve objavljene zadatke klikom na "Zadaci". Klikom na profilno ime može pregledati  profil i vidjeti  statistiku i osvojene pehare. Voditelji mogu stvoriti novo natjecanje ili novi zadatak klikom na "novo natjecanje" ili "novi zadatak". Korisnik se može odjaviti klikom na "odjavi se".

\begin{figure}[H]
	\includegraphics[scale=0.41]{dijagrami/DijagramStanja.png}
	\centering
	\caption{Dijagram stanja}
	\label{fig:dijagramStanja}
\end{figure}

\eject
%
\section{Dijagram aktivnosti}

Dijagram aktivnosti na slici \ref{fig:dijagramAktivnosti} prikazuje proces kreiranja novog natjecanja. Voditelj se prijavljuje u sustav te nakon uspješne prijave odabire opciju za organiziranje novog natjecanja. Web stranica preko baze podataka dohvaća dostupne zadatke te ih prikazuje kao dio forme u koju voditelj unosi podatke te odabire zadatke koji će se ispitivati u sklopu natjecanja. Nakon slanja unesenih podataka web aplikaciji, ona ih prosljeđuje bazi podataka koja ih zatim pohranjuje i šalje potvrdu o uspješnom stvaranju novog natjecanja. 
\\
\begin{figure}[H]
	\includegraphics[scale=0.5]{dijagrami/act_novo_natjecanje.png}
	\centering
	\caption{Dijagram aktivnosti - organiziranje novog natjecanja}
	\label{fig:dijagramAktivnosti}
\end{figure}
%
%\textbf{\textit{dio 2. revizije}}\\
%
%\textit{Potrebno je priložiti dijagram aktivnosti s pripadajućim opisom. Dijagram aktivnosti treba prikazivati značajan dio sustava.}
%
%\eject
\section{Dijagram komponenti}

Dijagram komponenti prikazan na slici \ref{fig:dijagramKomponenti} opisuje organizaciju i međuovisnost  komponenti, interne strukture i odnose prema okolini. Web aplikacija sastoji se od komponenti: Controllers, Services, Repositories, DTO i Models. Controllers pruža REST API sučelje na koje vanjski web preglednik može slati zahtjeve i primati odgovore pomoću JSON datoteka. Repositories pristupa SQL bazi podataka koja ostvaruje sučelje SQL API. Podaci koji su pristigli iz baze se šalju dalje MVC arhitekturi u obliku DTO (Data transfer object). Repositories upisuje podatke iz baze podataka u komponentu Services. Models oblikuje korištene entitete i njihov međuodnos. Controllers šalje i prima podatke od komponente Services. Controllers pruža MAIL API sučelje Mailjet preko kojeg šalje mailove i Judge0 API za evaluaciju programskog rješenja.
\\
\begin{figure}[H]
	\includegraphics[scale=0.5]{dijagrami/dijagramKomponenti.png}
	\centering
	\caption{Dijagram komponenti}
	\label{fig:dijagramKomponenti}
\end{figure}
%
%\textbf{\textit{dio 2. revizije}}\\
%
%\textit{Potrebno je priložiti dijagram komponenti s pripadajućim opisom. Dijagram komponenti treba prikazivati strukturu cijele aplikacije.}
