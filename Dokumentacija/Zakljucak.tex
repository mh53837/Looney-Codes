\chapter{Zaključak i budući rad}
		
\noindent Zadatak naše grupe bio je razvoj web aplikacije za provjeru riješenih programskih zadataka i
sudjelovanje u natjecanjima. Razvoj aplikacije trajao je nešto više od 3 mjeseca. Proces razvoja podijeljen je u dvije faze.

		U prvoj fazi fokusirali smo se na definiranje zahtjeva, okupljanje tima, podjelu tima na podtimove i izradu temeljne dokumentacije. Obrasci uporabe, sekvencijski dijagrami, dijagram razreda te model baze podataka pružili su jasnu viziju rješenja, olakšavajući rad podtimovima zaduženima za implementaciju (\textit{backend} i \textit{frontend}). 
		
		Druga faza bila je više fokusirana na razvoj web aplikacije. Članovi tima su samostalno radili na implementaciji rješenja, suočavajući se s izazovima vezanim uz odabrane alate i programskih jezika. Unatoč manjku iskustva u pojedinim područjima, članovi su uspješno savladali izazove, i uspjeli su ostvariti sve funkcionalnosti aplikacije. Dokumentacija, poput UML dijagrama i prateće dokumentacije, pridonijela je transparentnosti sustava, olakšavajući budućim korisnicima razumijevanje i prilagodbu. Temelji za aplikaciju, definirani u prvoj fazi, uštedjeli su vrijeme i pridonijeli efikasnosti rada članova.
		
		Komunikacija članova ostvarena je putem WhatsAppa i Discorda. Pridonijela je informiranosti tima o napretku projekta, lakšu i bržu suradnju članova tima, a zajednički rad na istom projektu bio je vrijedno iskustvo za sve sudionike. Unatoč postignućima, prepoznajemo prostor za daljnje usavršavanje aplikacije. Neka od mogućih unapređenja aplikacije su dodavanje grupa za natjecanja, dodavanje tagova za kategorije zadataka, izrada mobilne aplikacije. 
		  
		Sudjelovanje u razvoju aplikacije BytePit bilo je jedinstveno i iznimno vrijedno iskustvo svim članovima, donijelo je timu osjećaj postignuća, naglašavajući važnost koordinacije i organizacije u zajedničkom projektu. Unatoč izazovima, zadovoljni smo rezultatima i radujemo se daljnjem usavršavanju ove platforme.
		\eject 