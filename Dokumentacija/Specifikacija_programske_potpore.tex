\chapter{Specifikacija programske potpore}
		
	\section{Funkcionalni zahtjevi}
			
			\noindent \textbf{Dionici:}
			
			\begin{packed_enum}
				
				\item Voditelji
				\item Natjecatelji				
				\item Administrator
				\item Razvojni tim
				
			\end{packed_enum}
			
			\noindent \textbf{Aktori i njihovi funkcionalni zahtjevi:}
			
			
			\begin{packed_enum}
				\item  \underbar{Neregistrirani/neprijavljeni korisnik (inicijator) može:}
				
				\begin{packed_enum}
					
					\item vidjeti kalendar s budućim natjecanjima
					\item pregledati dostupne zadatke na stranici
					\item pregledati profile natjecatelja i voditelja
					\item registrirati se u sustav stvaranjem novog korisničkog računa pri čemu odabire jednu od uloga (natjecatelj ili voditelj), a potrebni su mu: korisničko ime, fotografija, lozinka, ime, prezime te email adresa
					
				\end{packed_enum}
			
				\item  \underbar{Natjecatelj (inicijator) može:}
				
				\begin{packed_enum}
					
					\item sudjelovati na natjecanju
					\item vidjeti rang listu natjecatelja na natjecanju kojeg je i sam bio sudionik
					\item vidjeti popis svih učitanih rješenja od ostalih sudionika za prethodno završena natjecanja
					\item pristupiti rješavanju već objavljenih zadataka
					\item izraditi virtualno natjecanje odabirom nekog prošlog natjecanja ili nasumičnim generiranjem zadataka 
					\item vidjeti vlastiti profil s osobnim podacima, statistikom o broju točno riješenih zadatka, broju isprobanih zadataka te prikaz pehara za osvojena natjecanja
					
				\end{packed_enum}
				
				\item  \underbar{Aktivni natjecatelj (inicijator) može:}
				
				\begin{packed_enum}
					
					\item rješavati zadatke i slati datoteke s programskim kodom tijekom natjecanja na kojem se natječe 
					\item osvojiti bodove na natjecanju s obzirom na potrošeno vrijeme za rješavanje zadatka i postotak točnih primjera 
					\item na temelju postignuća za prva tri mjesta osvojiti pehar koji je vidljiv na vlastitom profilu
					
				\end{packed_enum}
				
				\item  \underbar{Voditelj (inicijator) može:}
				
				\begin{packed_enum}
					
					\item pregledati dostupne zadatke na stranici
					\item izraditi novi zadatak pri čemu treba definirati: naziv zadatka, broj bodova, vremensko ograničenje, tekst zadatka i primjere za evaluaciju
					\item organizirati novo natjecanje pri čemu treba definirati: vrijeme početka i završetka, broj zadataka, koji zadaci će biti aktivni te po želji sličicu pehara
					\item uređivati vlastite prethodno objavljene zadatke te natjecanja
					\item vidjeti vlastiti profil s osobnim podacima, popisom učitanih zadataka s mogućnošću sortiranja te kalendar s popisom objavljenih natjecanja
										
				\end{packed_enum}
				
				\item  \underbar{Administrator (inicijator) može:}
				
				\begin{packed_enum}
					
					\item uređivati sve zadatke i natjecanja
					\item potvrditi voditelja prilikom registracije
					\item vidjeti popis svih registriranih korisnika i njihovih osobnih podataka
					\item mijenjati dodijeljena prava i osobne podatke

				\end{packed_enum}
				
				\item  \underbar{Baza podataka (sudionik) može:}
				
				\begin{packed_enum}
					
					\item pohranjuje sve podatke o korisnicima i njihovim ovlastima te zadacima i natjecanjima
					\item pohranjuje rezultate natjecanja, rješenja zadataka i statistiku natjecatelja			
					
				\end{packed_enum}
			\end{packed_enum}
			
			\eject 
			
			
				
			\subsection{Obrasci uporabe}
				
				\textbf{\textit{dio 1. revizije}}
				
				\subsubsection{Opis obrazaca uporabe}
					\textit{Funkcionalne zahtjeve razraditi u obliku obrazaca uporabe. Svaki obrazac je potrebno razraditi prema donjem predlošku. Ukoliko u nekom koraku može doći do odstupanja, potrebno je to odstupanje opisati i po mogućnosti ponuditi rješenje kojim bi se tijek obrasca vratio na osnovni tijek.}\\
					

%					\noindent \underbar{\textbf{UC$<$broj obrasca$>$ -$<$ime obrasca$>$}}
%					\begin{packed_item}
%	
%						\item \textbf{Glavni sudionik: }$<$sudionik$>$
%						\item  \textbf{Cilj:} $<$cilj$>$
%						\item  \textbf{Sudionici:} $<$sudionici$>$
%						\item  \textbf{Preduvjet:} $<$preduvjet$>$
%						\item  \textbf{Opis osnovnog tijeka:}
%						
%						\item[] \begin{packed_enum}
%	
%							\item $<$opis korak jedan$>$
%							\item $<$opis korak dva$>$
%							\item $<$opis korak tri$>$
%							\item $<$opis korak četiri$>$
%							\item $<$opis korak pet$>$
%						\end{packed_enum}
%						
%						\item  \textbf{Opis mogućih odstupanja:}
%						
%						\item[] \begin{packed_item}
%	
%							\item[2.a] $<$opis mogućeg scenarija odstupanja u koraku 2$>$
%							\item[] \begin{packed_enum}
%								
%								\item $<$opis rješenja mogućeg scenarija korak 1$>$
%								\item $<$opis rješenja mogućeg scenarija korak 2$>$
%								
%							\end{packed_enum}
%							\item[2.b] $<$opis mogućeg scenarija odstupanja u koraku 2$>$
%							\item[3.a] $<$opis mogućeg scenarija odstupanja  u koraku 3$>$
%							
%						\end{packed_item}
%					\end{packed_item}
					
					\noindent \underbar{\textbf{UC1 - Pregled kalendara}}
					\begin{packed_item}
						
						\item \textbf{Glavni sudionik: } Korisnik
						\item \textbf{Cilj:} Pregledati kalendar s nadolazećim natjecanjima 
						\item \textbf{Sudionici:} Baza podataka
						\item \textbf{Preduvjet:} -
						\item \textbf{Opis osnovnog tijeka:}
						
						\item[] \begin{packed_enum}
							\item Korisnik otvara početnu stranicu web aplikacije
							\item Prikazuje se kalendar
							\item Korisnik odabire određeni datum
							\item Aplikacija prikazuje popis nadolazećih natjecanja za odabrani datum
							\item Korisnik odabire natjecanje
							\item Prikazuju se detalji i informacije o odabranom natjecanju
						\end{packed_enum}
					\end{packed_item}
					
					
					\noindent \underbar{\textbf{UC2 - Pregled zadataka}}
					\begin{packed_item}
						
						\item \textbf{Glavni sudionik: }Korisnik
						\item \textbf{Cilj:} Pregledati zadatke završenih natjecanja
						\item \textbf{Sudionici:} Baza podataka
						\item \textbf{Preduvjet:} -
						\item \textbf{Opis osnovnog tijeka:}
						
						\item[] \begin{packed_enum}
							\item Korisnik odabire završeno natjecanja
							\item Prikazuje se lista zadatka povezana s odabranim natjecanjem
							\item Korisnik odabire specifičan zadatak
							\item Aplikacija prikazuje detalje zadatka
						\end{packed_enum}
					\end{packed_item}
					
					
					\noindent \underbar{\textbf{UC3 - Pregled profila natjecatelja i voditelja}}
					\begin{packed_item}
						
						\item \textbf{Glavni sudionik: }Korisnik
						\item \textbf{Cilj:} Pregledati zadatke završenih natjecanja
						\item \textbf{Sudionici:} Baza podataka
						\item \textbf{Preduvjet:} -
						\item \textbf{Opis osnovnog tijeka:}
						
						\item[] \begin{packed_enum}
							\item Korisnik otvara profil određenog korisnika klikom na njihovo korisničko ime
							\item Ako je otvoren profil natjecatelja, aplikacija prikazuje informacije o broju točno riješenih zadataka, broju isprobanih zadataka te osvojene pehare
							\item Ako je otvoren profil voditelja, aplikacija prikazuje popis učitanih zadataka i kalendar s popisom objavljenih natjecanja
						\end{packed_enum}
					\end{packed_item}
					
					
					
					
					\noindent \underbar{\textbf{UC4 - Registracija}}
					\begin{packed_item}
						
						\item \textbf{Glavni sudionik: }Korisnik
						\item  \textbf{Cilj:} Stvoriti korisnički račun
						\item  \textbf{Sudionici:} Baza podataka, Administrator 
						\item  \textbf{Preduvjet:} -
						\item  \textbf{Opis osnovnog tijeka:}
						
						\item[] \begin{packed_enum}
							
							\item Korisnik odabire opciju za registraciju 
							\item Korisnik popunjava obrazac za registraciju s potrebnim podacima 
							\item Korisnik prima obavijest o uspješnoj registraciji na svoju e-mail adresu 
							\item Ako je korisnik odabrao ulogu "voditelj", sustav šalje obavijest administratoru 
							\item Administrator potvrđuje registraciju voditelja 
						\end{packed_enum}
						
						\item  \textbf{Opis mogućih odstupanja:}
						\item[] \begin{packed_item}
							
							\item[2.a] Već zauzeto korisničko ime ili e-mail adresu, unesni podaci u nedozvoljenom formatu ili neispravna e-mail adresa 
							\item[] \begin{packed_enum}
								
								\item Sustav obavještava korisnika o neuspjelom unosu i prikazuje relevantne poruke o greškama 
								\item Korisnik mijenja potrebne podatke te završava unos ili odustaje od registracije 
								
							\end{packed_enum}
							
							\item[5.a] Administrator odbija zahtjev za registraciju voditelja natjecanja:
							\item[] \begin{packed_enum}
								
								\item Sustav obavještava korisnika putem e-maila
								
							\end{packed_enum}
						\end{packed_item}
					\end{packed_item}
					
					\noindent \underbar{\textbf{UC5 - Prijava}}
					\begin{packed_item}
						
						\item \textbf{Glavni sudionik: }Korisnik
						\item \textbf{Cilj:} Dobiti pristup korisničkom sučelju
						\item \textbf{Sudionici:} Baza podataka
						\item \textbf{Preduvjet:} -
						\item \textbf{Opis osnovnog tijeka:}
						\item[] \begin{packed_enum}
							\item Unos korisničkog imena i lozinke
							\item Potvrda o ispravnosti unesenih podataka
							\item Pristup korisničkim funkcijama
						\end{packed_enum}
						
						\item  \textbf{Opis mogućih odstupanja:}
						
						\item[] \begin{packed_item}
							
							\item[2.a] Neispravno korisničko ime/lozinka
							\item[] \begin{packed_enum}
								
								\item Sustav obavještava korisnika o neuspjelom upisu i vra´ca ga na stranicu za registraciju
								
							\end{packed_enum}
						\end{packed_item}
					\end{packed_item}
					
					\noindent \underbar{\textbf{UC6 - Pregled profila}}
					\begin{packed_item}
						
						\item \textbf{Glavni sudionik: }Korisnik
						\item \textbf{Cilj:} Pregledati vlastiti profil
						\item \textbf{Sudionici:} Baza podataka
						\item \textbf{Preduvjet:} -
						\item \textbf{Opis osnovnog tijeka:}
						\item[] \begin{packed_enum}
							\item Korisnik pristupa opciji "Moj profil" klikom na ikonu profila
							\item Aplikacija prikazuje podatke o korisniku (ime, fotografija, lozinka, ime, prezime i e-mail adresa)
						\end{packed_enum}
					\end{packed_item}
					
					
					
					\noindent \underbar{\textbf{UC7 - Pregled vlastite statistike}}
					\begin{packed_item}
						
						\item \textbf{Glavni sudionik: }Korisnik (natjecatelj)
						\item \textbf{Cilj:} Pregledati vlastitu statistiku unutar web aplikacije
						\item \textbf{Sudionici:} Baza podataka
						\item \textbf{Preduvjet:} Korisnik je prijavljen
						\item \textbf{Opis osnovnog tijeka:}
						
						\item[] \begin{packed_enum}
							\item Korisnik pristupa svojem profilu i odabire opciju "Moje statistike".
							\item Aplikacija prikazuje statistike korisnika(broj točno riješenih zadataka, broju isprobanih zadataka, pehare za osvojena natjecanja)
						\end{packed_enum}
					\end{packed_item}

					
					
				
					
				\subsubsection{Dijagrami obrazaca uporabe}
					
					\textit{Prikazati odnos aktora i obrazaca uporabe odgovarajućim UML dijagramom. Nije nužno nacrtati sve na jednom dijagramu. Modelirati po razinama apstrakcije i skupovima srodnih funkcionalnosti.}
				\eject		
				
			\subsection{Sekvencijski dijagrami}
				
				\textbf{\textit{dio 1. revizije}}\\
				
				\textit{Nacrtati sekvencijske dijagrame koji modeliraju najvažnije dijelove sustava (max. 4 dijagrama). Ukoliko postoji nedoumica oko odabira, razjasniti s asistentom. Uz svaki dijagram napisati detaljni opis dijagrama.}
				\eject
	
		\section{Ostali zahtjevi}
		
			\textbf{\textit{dio 1. revizije}}\\
		 
			 \textit{Nefunkcionalni zahtjevi i zahtjevi domene primjene dopunjuju funkcionalne zahtjeve. Oni opisuju \textbf{kako se sustav treba ponašati} i koja \textbf{ograničenja} treba poštivati (performanse, korisničko iskustvo, pouzdanost, standardi kvalitete, sigurnost...). Primjeri takvih zahtjeva u Vašem projektu mogu biti: podržani jezici korisničkog sučelja, vrijeme odziva, najveći mogući podržani broj korisnika, podržane web/mobilne platforme, razina zaštite (protokoli komunikacije, kriptiranje...)... Svaki takav zahtjev potrebno je navesti u jednoj ili dvije rečenice.}
			 
			 
			 
	