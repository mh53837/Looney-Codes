\chapter{Dnevnik promjena dokumentacije}
		
		\textbf{\textit{Kontinuirano osvježavanje}}\\
				
		
		\begin{longtblr}[
				label=none
			]{
				width = \textwidth, 
				colspec={|X[2]|X[13]|X[3]|X[3]|}, 
				rowhead = 1
			}
			\hline
			\textbf{Rev.}	& \textbf{Opis promjene/dodatka} & \textbf{Autori} & \textbf{Datum}\\[3pt] \hline
			0.1 & Preuzet i postavljen predložak.	& Nikola Vlahović & 20.10.2023. 		\\[3pt] \hline 
			0.2	& Dodan opis projektnog zadatke.\newline Upisani članovi. & Marko Varga & 21.10.2023. 	\\[3pt] \hline
   			0.3 & Nadopisan opis projektnog zadatka.	& Marina Hrbud & 22.10.2023. 		\\[3pt] \hline 
   			0.4 & Dodani funkcionalni zahtjevi	& Lara Marčec & 23.10.2023. 		\\[3pt] \hline 
   			0.5 & Dodani opisi obrazaca uporabe	& Nikola Vlahović, Lara Marčec & 29.10.2023. 		\\[3pt] \hline
   			0.6 & Dodani dijagrami obrazaca uporabe & Jakov Novak, Marko Varga & 29.10.2023. \\[3pt] \hline
		\end{longtblr}
	
	
		\textit{Moraju postojati glavne revizije dokumenata 1.0 i 2.0 na kraju prvog i drugog ciklusa. Između tih revizija mogu postojati manje revizije već prema tome kako se dokument bude nadopunjavao. Očekuje se da nakon svake značajnije promjene (dodatka, izmjene, uklanjanja dijelova teksta i popratnih grafičkih sadržaja) dokumenta se to zabilježi kao revizija. Npr., revizije unutar prvog ciklusa će imati oznake 0.1, 0.2, …, 0.9, 0.10, 0.11.. sve do konačne revizije prvog ciklusa 1.0. U drugom ciklusu se nastavlja s revizijama 1.1, 1.2, itd.}
